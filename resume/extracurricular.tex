

\cvsection{Projects \& Experience}
\vspace{-0.5\baselineskip}
\begin{cventries}
  \cventry
    {Open AI, Langchain, Pinecone DB, PostgreSQL, Node.JS, ts.ED, React, Azure}
    {Generative AI Quiz/Journey builder}
    {Preezie, Melbourne}
    {Sep. 2023 - Oct. 2023}
    {
      \begin{cvitems}
        \item {Main backend developer for a Generative AI POC project investigating the rise of AI generated content and startup competitors.}
        \item {Learned new prompt engineering techniques, Vector search using Open AI Embeddings & various new skills. Solution was one of the many demonstrated by the CEO to investors for potential Series B funding.}
      \end{cvitems}
    }
  \cventry
    {React, Next.JS, HTML5, TypeScript, Tailwindcss, CSS3, Google Firebase, LaTeX, Node.JS, APIs }
    {www.YasirFayrooz.com}
    {Melbourne}
    {Aug. 2022 - Present}
    {
      \begin{cvitems}
        \item {Developed a terminal themed portfolio website to refresh my React skills in my own time. Integrated various APIs eg: compiling my resume using the typesetting language LaTeX. Continuing to build and support the app over time.}
        \item {Learned TailwindCSS, Next.JS and hosted my first personal website on Google Firebase.}
      \end{cvitems}
    }
  \cventry
    {C\#, .NET 7, React/Angular, Postgres, DDD, SQL Server, CosmosDb, Elastic Search, Redis Cache, Azure Service Bus, MediatR, CAP}
    {.NET Developer - Preezie}
    {Preezie, Melbourne}
    {Aug. 2022 - Present}
    {
      \begin{cvitems}
        \item {Developing a new improved solution based on asynchronous event-driven microservices, domain driven design and clean architecture practices while maintaining the existing legacy application.}
        \item {Contributing mainly to backend development. Also assisting in some micro-frontend development in both Angular and React.}
        \item {Using best practices and various design principals and patterns to ensure efficiency when scaling.}
	  \end{cvitems}
    }
  \cventry
    {Azure Cosmos DB, SQL database, Blob storage, VNET, App services, Media services, Duende Identity Server 4, .NET 5 APIs, Angular, NLog, Azure devops, Automated CI/CD Pipelines}
    {Public Safety Management - An Azure Cloud based solution}
    {Motorola Solutions, Melbourne}
    {Oct. 2021 - Aug. 2022}
    {
      \begin{cvitems}
        \item {Co-developed a full stack solution for CrimeStoppers Tasmania's online crime reporting website with future support for multi-tenancy. Achieved media press recognition and awards.}
        \item {A site accessible to the public to report crimes anonymously and upload media. Fully mobile responsive.}
        \item {Near real-time management portal to process reports submitted by the public and stream media backed by an identity server.}
        \item {Features such as searching/filtering/pagination, audit logs, permissions, downloading, report statistics implemented in the portal.}
      \end{cvitems}
    }
  \cventry
    {VBA, Winforms, SQL database, SMTP server, Google OAuth, Angular 13, .NET 6, MJML}
    {NOCC Notification App}
    {Motorola Solutions, Melbourne}
    {May. 2021 - Aug. 2022}
    {
      \begin{cvitems}
        \item {Maintained and developed a legacy product by an ex-Motorola employee. The project sends email notifications to customers in the case of an outage to their service.}
        \item {Reworked the app months later into a modern web application with a fine grained permission system backed by Google account authentication. Increased productivity and decreased the need of supporting the application vastly.}
      \end{cvitems}
    }
  \cventry
    {Python, GIT, Markdown, AI}
    {Artifical Intelligence: Pacman AI CTF Contest}
    {RMIT University}
    {Oct. 2020}
    {
      \begin{cvitems}
        \item {Developed a Pacman AI using numerous techniques such as A* Search, Expectimax \& Reinforcement Learning/Q-Learning with function approximation in a group of 3 team members (inclusive).}
        \item {Finished at 3rd place out of 74 teams total for best AI (groups of 3-4 in each team).}
      \end{cvitems}
    }
\end{cventries}
